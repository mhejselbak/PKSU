\documentclass[12pt]{article}
\usepackage{amsmath} % flere matematikkommandoer
\usepackage{amssymb} % mere matematik stads
\usepackage[utf8]{inputenc} % æøå
\usepackage[T1]{fontenc} % mere æøå
\usepackage[danish]{babel} % orddeling
\usepackage{verbatim} % så man kan skrive ren tekst
\usepackage[all]{xy} % den sidste (avancerede) formel i dokumentet

\begin{document}
\section{Review}
\subsection{A rational design process: How and why to fake it.}

Artiklen handler om den ideelle måde at designe programmer på. Den er delt i 3 nogenlunde klare dele. Første del omhandler den ideelle rationale designproces og hvorfor den ikke kan opnås. Anden del er forfatternes eksempel på en designproces. Og trejde del omhandler styrken af forfatternes designproces mod nutidige metoder.\\\\
Den ideelle designproces\\
Artiklen starter ud med at fortælle om den ideelle rationale designproces, hvor en designproces næsten kan sammenlignes med et matematiske bevis, hvor i man kan følge en logisk tråd fra start til slut. Herefter klargører den hvorfor dette stort ser ikke er en mulig måde at udføre designprocessen på. Artiklen argumentere at de logiske skridt ikke kan følges f.eks pga. kravende til programmet kan ændre sig løbende eller at menneskelige fejl opstår. Efterfølgende forklares hvorfor det er god ide at tilnærme sig en sig en sådan designproces, selvom den ikke er mulig at opnå.\\\\
Designproces eksemplet\\
Artiklen udlægger herefter et eksempel på en rationel designproces man kan bruge. Gennem 7 punkter forklares der hvordan man skal dokumentere og sørge for at kravende er gode for designprocessen, hvordan man skal bryde programmet ned i moduler som er overkommelige og sørge for modulernes indbyrdes funktioner. Det forklares at stort set hele processen er dokumentation som klargører hvad der skal gøres, og at det derfor er nemmere at lave de logiske rationalle skridt som skal tages.\\\\
Forfatternes proces mod nutidige metoder\\
Til sidst forklarer artiklen om nutidig dokumentation og problemerne ved det. Nutidig dokumentation lider under at programmøre betragter dokumentation som noget overflødigt, og derfor er dokumentationen ukomplet og upræcis. Derefter slås det fast hvor vigtig god dokumentation er for den ideelle designprocess.\\\\
Det vi i gruppen mener der er værd at tage med fra denne artikel er selve ideen om at bruge en designproces, som nærmer sig den logiske og rationelle proces der er beskrevet i starten af artiklen. Hvis man kan finde en nogenlunde standart måde at design sine programmer er det helt klart fordelagtigt. Eksemplet der er beskrevet virker meget gennemtænkt og brugbart. Designprocessen foreslået i artiklen ligner meget den proces som vi skal bruge i vores delrapporter. Det ligner også meget det som er blevet gennemgået i vores undervisning og vi forsøger at følge den efter bedste formåen.
\end{document}