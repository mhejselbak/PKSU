\documentclass[12pt]{article}
\usepackage{amsmath} % flere matematikkommandoer
\usepackage[utf8]{inputenc} % æøå
\usepackage[T1]{fontenc} % mere æøå
\usepackage[danish]{babel} % orddeling
\usepackage{verbatim} % så man kan skrive ren tekst
\usepackage[all]{xy} % den sidste (avancerede) formel i dokumentet
\usepackage{graphicx}
\usepackage{listings}
\usepackage{url}
\title{ProjektKursus Systemudvikling 2014\\Delrapport 2}
\author{}

\begin{document}
\maketitle
\url{www.kennethchristensen.dk} Her kan man se vores projekt.\\
Gruppemedlemmer:\\
Kenneth Christensen: 02 08 93\\Michael Jensen: 01 07 93\\Rune Pedersen: 01 11 82\\Rasmus Hansen: 03 12 92
\\\\
Instructor: Kasper Passov

\pagebreak
\section{Abstract}
Vi har valgt at udarbejde et projekt for  ”BMX-Butikken”, det er en butik der har specialiseret sig indenfor BMX sporten, som er placeret i københavn.  De sælger alt indenfor BMX, som indebærer alt fra tøj og sko, til cykler og dele. Udover en fysisk butik på nørrebrogade, har bmxbutikken også en online webshop som ligger på www.bmxbutikken.dk.
Projektet går ud på at lave en applikation til smartphones. Denne skal hjælpe bmx kørerer med at finde nye steder at køre på, som før var forbeholdt de lokale, derudover vil applikationen også vise  brugerne hvor bmx-butikken ligger og eventuelle samarbejdspartnere.
Ideen med applikationen er at den skal appelere til bmx kørere, dette kan både være lokale som vil udforske deres by yderligere, eller folk fra andre dele af landet på besøg i København samt såkaldte bmx eller skate turister. Applikationen Vil hovedsageligt dreje sig om københavnsområdet.
Vi vil bygge appen op ved hjælp af javascript, html og css. Det kommer til at fungere som en hjemmeside, som appen blot kommer til at åbne i en mobil venlig udgave.

\section{IT-projektets formål og rammer}
Følgende model er en FACTOR analyse af projektet og giver en kort og konkret beskrivelse af rammerne for projektet.
\begin{figure}[h]
    \includegraphics[scale=0.5]{factor.png}
\end{figure}

\pagebreak

\section{Kravspecifikation for IT-løsningen}
\subsection*{Funktionelle krav}
\textbf{Add spot} \\ Tilføjer et spot til Databasen hvor der måske kan være et billede med.\\
\textbf{Menu}\\ Indeholder 5 ting du kan trykke på der kan vejlede bruger: Spots, shops, settings, Add spot, Search\\
\textbf{Streetspots}\\ Viser brugeren en liste over de forskellige spots.\\
\textbf{Markers}\\ Viser på kortet 3 ting: Brugerens lokation, spots og bmxbutikkens lokation\\
\textbf{Rutevejledning}\\ Vejleder brugeren til det pågældende punkt på kortet ved hjælp af google maps.\\
\textbf{Bruger lokation}\\ Finder brugerens lokation.\\
\textbf{Shops}\\ Centrerer kortet på bmxbutikken.
\subsection*{Ikke funktionelle krav:}
\textbf{Sprog}\\ Siden skal være på engelsk, så den kan bruges af turister, men samtidig af de lokale da de fleste i miljøet snakker engelsk.\\
\textbf{Bruger venligt UI}\\ Nemt og intuitivt for brugeren at komme rundt.\\
\subsection*{Constraints}
Systemet skal være kompatibelt med både computere og telefoner.[Implementation requirement]\\
Kortet skal vise en marker ved bmxbutikken[User Interface Requirement]\\
Koden skal være skrevet i php/javascript/html/mysql.[Implementation requirement]\\
\pagebreak\\

(b) En use case model, der beskriver system-funktionaliteten. Der ønskes et højniveau-diagram,
som giver overblik over hvilke use cases de forskellige aktører har.\\
\begin{figure}[htb]
\begin{center}
\includegraphics[scale = 0.75]{usecasemodel}
\end{center}
\end{figure}

Højniveau diagram der skal illustrere de forskellige aktører\\\\
Som det ses er der 2 aktører, brugeren og Ejeren. Brugeren kan benytter de forskellige funktioner i appen, og ejeren redigere og kontrollere uploads og spots. 
\pagebreak\\
(c) Tre specificerede use cases, som er særlig vigtige i jeres system, se fx OOSE figur 4-22.\\
\setlength\parindent{0pt}
\section*{Use-cases}
\subsection*{Få information om et spot}
\hrule\vspace{5mm}
\textit{Use case name:} FindSpot\\
\hrule\vspace{5mm}
\textit{Actor:} en bruger der gerne vil finde information om et spot\\
\hrule\vspace{5mm}
\textit{Entry condition:} brugeren åbner appen\\
\hrule\vspace{5mm}
\textit{Flow of events:}
\begin{enumerate}
\item Brugeren bliver præsenteret for et kort med sin egen positition i centrum og en menu linje i toppen
\item Brugeren trykker på en "marker" på kortet for det spot han vil se information.
\end{enumerate}
\hrule\vspace{5mm}
\textit{Exit condition:} Brugeren ser nu et pop-up vindue med information om spottet.\\
\hrule\vspace{5mm}
\newpage
\subsection*{Tilføj spot}
\hrule\vspace{5mm}
\textit{Use case name:} AddSpot\\
\hrule\vspace{5mm}
\textit{Actor:} en bruger der gerne vil tilføje et spot\\
\hrule\vspace{5mm}
\textit{Entry condition:} brugeren åbner appen\\
\hrule\vspace{5mm}
\textit{Flow of events:}
\begin{enumerate}
\item Brugeren trykker på knappen "Add spot", der sender ham videre til en udfyldelses formular
\item Brugeren udfylder formularen og gemmer denne, formularen sendes derefter til godkendelse hos administratoren
\end{enumerate}
\hrule\vspace{5mm}
\textit{Exit condition:} Brugeren sendes retur til forsiden\\
\hrule\vspace{5mm}
\newpage
\subsection*{Ændre indstillinger}
\hrule\vspace{5mm}
\textit{Use case name:} ChangeCondition\\
\hrule\vspace{5mm}
\textit{Actor:} en bruger der gerne vil ændre indstillingerne\\
\hrule\vspace{5mm}
\textit{Entry condition:} brugeren åbner appen\\
\hrule\vspace{5mm}
\textit{Flow of events:}
\begin{enumerate}
\item Brugeren trykker på indstillinger, der sender brugeren videre til en ny side
\item Brugeren bliver præsenteret for de forskellige indstillinger han kan vælge
\item Når brugeren han ændret indstillinger er der en ok knap som tilføjer indstillingerne
\end{enumerate}
\hrule\vspace{5mm}
\textit{Exit condition:} Indstillingerne er nu i funktion og brugeren sendes tilbage til forsiden\\
\hrule\vspace{5mm}
\pagebreak

(d) Et klassediagram over jeres problemområde (solution-domain).\\\\
\textbf{HJÆLP KASPER!!!!!}\\\\
(e) Sekvens-diagrammer over de 3 use-cases specificeret i punkt (c).\\
Husk at alle diagrammer skal være fulgt af tekstbeskrivelser, der gør diagrammerne fuldt
forståelige også for læsere uden særligt domænekendskab.\\
\begin{figure}[h]
\includegraphics[scale = 0.5]{sekdia1}
\end{figure}

Sekvensdiagram over find spot information:\\
Brugeren starter App'en på sin telefon. App'en viser kort interfacen og brugeren vælger et spot fra kortet. 
App'en henter spot informationerne fra database og viser dem i et overlay.
\newpage

\begin{figure}[h]
\includegraphics[scale = 0.5]{sekdia2}
\end{figure}

Sekvensdiagram over add spot funktionen:\\
Brugeren starter App'en på sin telefon. App'en viser kort interfacen og brugeren vælger add spot fra menuen.
App'en viser en add spot formular. Brugeren udfylder formularen med spot informationerne og submiter den.
App'en sender spotet til databasen.

\newpage

\begin{figure}[h]
\includegraphics[scale = 0.5]{sekdia3}
\end{figure}

Sekvensdiagram over indstillinger:\\
Brugeren starter App'en på sin telefon. App'en viser kort interfacen og brugeren vælger settings fra menuen.
App'en viser en radius slider. Brugeren vælger en radius og og submiter den. App'en gemmer den nye radius.


\pagebreak

\section{Systemdesign sammenfatning}
Kapitlet resumerer jeres foreløbige system-design så kort og klart som muligt. Samtidig
udpeger I de vigtigste udestående design- og implementationsopgaver.\\

Vores plan vedrørende udviklingen af appen har indtil nu, været at bygge appen op i java med android developer tools. Det gav os en del udfordring, da vi ikke kunne få den simpleste udgave til at køre på en simulator. De mange timer og forsøg har gjort at vi følte os nødsaget til at gøre projektet mindre omfattende. 
\\\\
Derfor har vi valgt at bygge applikationen op webbaseret i html, css, javascript og mySQL, hvilket er programmer og metoder vi har erfaring med i forvejen. 
Hele implementeringen af Google maps foregår med javascript(altså som en hjemmeside), hvor vi så bagefter vil lave en android webview app, så vi kan tilføje applikationen på google play store. vi mangler at implementere alt andet end kortet på nuværende tidspunkt.

\section{Program- og systemtest}
Dokumenter jeres foreløbige test af IT-løsningen. I kapitlet sammenfattes hovedresultaterne af
jeres test-aktiviteter; mens test plan, test case specification, test incident report og test report
summary placeres som bilag.\\
Vi har tænkt os at teste det ved at få nogen mennesker til at lave think-aloud mens de prøver hjemmesiden. Dette kommer snart til at ske og vil være med i delrapport 3. Derudover vil vi selfølgelig selv teste løbenede på diverse funktioner som vi får lavet til hjemmesiden for at sikre os at det kører som det skal.
\pagebreak
\section{Brugergrænseflade og interaktionsdesign}
(a) Præsenter skærmbilleder af de mest interessante dele af jeres brugergrænseflade.

\begin{figure}[h]
\includegraphics[scale = 0.3]{screen1}
\end{figure}

Billedet viser vores foreløbige hjemmeside der blot indeholder en menu uden funktioner, samt et kort med "spots" på den måde som vi vil præsentere dem. \\ \\
Settings: 
Når man trykker på settings kommer der en slider, som der kan bestemme hvor stor en radius siden skla loade spots fra, vurderet fra brugerens nuværende position.\\\\
Spots: 
Når du trykker her kommer der en liste frem over alle de forskellige spots som man herefter trykke på dem. Dette vil centrerer dem på kortet.\\\\
Shops: 
Centrerer kortet på BMXButikken.\\\\
Add spot: 
Når der trykkes på dette kommer der en formular frem som man skal udfylde med: Name, Address, City, zip og billede.\\
Search: Kommer frem med et søgefelt hvor man kan søge på spots via navn/zipcode/by.\\

(b) Illustrer flowet/dynamikken i brugerinteraktionen mellem skærmbillederne.\\
Dette vil blive udfyldt i delrapport 3 eller 4 hvor vi er kommet længere med programmet.\\\\
(c) En audio-visuel præsentation af brugergrænsefladen af den seneste kørende prototype.\\\\
(d) Resultatet af seneste tænke-højt forsøg gennemført med een eller flere af jeres brugere.\\\\
Dette punkt vil blive udfyldt til delrapport 3 da vi har aftale med flere brugere.\\

\pagebreak
\section{Versionstyring}
Til aflevering: Som bilag skal vedlægges jeres nuværende commit-log samt jeres
programkode. Kommentér kort (ca 1/2 side) de vigtigste ændringer, der er sket i programkoden.\\

Det første udkast til systemet blev skrevet med java/google developer tools som vi havde planlagt det. Men efter mange timers forsøg, måtte vi erkende det måske var en uoverkommelig opgave.\\
Vi er derfor gået over til at lave appen webbaseret og bruge javascript, css, php, html og mySQL til at snakke med en database. Vi har nu fået det opdateret så den finder spots fra databasen. \\
\\
Derudover er vi også gået igang med at lave en app til android som der loader hjemmesiden ned så folk også kan bruge som app hvis de vil det. Denne vil også loade mappet så det ligger på deres telefon så de i princippet også kan bruge kortet uden net, som vil gøre det muligt for dem stadig at se alle spots men ikke kunne tilføje og ændre på indstillinger.




\pagebreak
\section{Projektsamarbejdet}
Til aflevering: Beskriv konkret og oplysende hvordan det går med samarbejdet med brugerne
og med arbejdet internt i gruppen. Herunder skal bl.a. oplyses antallet af møder med brugerne under
projektforløbet (fx på en tidslinje), mødeformen i gruppen, samt hvorledes jeres referat- og
dokumentationsform fungerer. Hvorledes prioriterer og styrer I projektindsatsen, så I sikrer
fremdrift på de felter, som er mest risikable/afgørende for et succesfuldt resultat? Herunder, beskriv
og diskuter:\\

(a) Hvad går godt?\\
Det interne arbejde i gruppen fungerer godt. Alle medlemmer i gruppen deltager aktivt, engageret og møder til tiden på de aftalte møde dage. Selve arbejdet prøver vi at fordele imellem os alt efter interesse og det sikrer en god stabil arbejds indsats.\\

(b) Hvad går mindre godt?\\

Vi har haft uventede problemer hvad angår systemdesignet da vi har måttet ændre kurs og gå over i en webbaseret udgave. Det har betydet at vi har haft nogle arbejdsspildtimer og vi er derfor ikke noget helt så langt som planlagt.\\
KENNETH SKAL SLUKKE SIN TELEFON.\\
(c) Hvad vil I gøre for at effektivisere jeres udviklingsarbejde?\\

I det vi tog skridtet fra android app til webbaseret app, har vi taget de første skridt i effektiviseringen. Derudover regner vi med at kunne begynde at arbejde selvstændigt på enkelte dele af systemet når vi har fået den første prototype helt op at køre. Det skullle bevirke at det bliver nemmere for de enkelte individer i gruppen at få plads i deres skema, til at få arbejdet på projektet og dermed få arbejdet flere timer. 

\pagebreak
\section{Review}

\section{Bilag}

\subsection{Commitlog}

\end{document}