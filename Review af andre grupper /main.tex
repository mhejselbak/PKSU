\documentclass[a4paper]{article}

\usepackage[english]{babel}
\usepackage[utf8]{inputenc}
\usepackage{amsmath}
\usepackage{graphicx}
\usepackage[colorinlistoftodos]{todonotes}

\title{Review 2}

\author{Gruppemedlemmer:\\
Kenneth Christensen: 02 08 93\\Michael Jensen: 01 07 93\\Rune Pedersen: 01 11 82\\Rasmus Hansen: 03 12 92
\\\\
Instruktør: Kasper Passov}

\begin{document}
\maketitle
\pagebreak
\section*{Reviews}
\subsection*{Designing for usability: key principles and what designers think}
Gruppens referat af artiklen er lidt kort, men tankerne beskrevet derefter dækker meget fint. Gruppen hævder herefter allerede at have fulgt artiklens tanker, sammen med deres kunde, men de forklarer ikke hvordan.
\subsection*{A rational design process: How and why to fake it.}
Til denne artikel er gruppens referat meget bedre og de dækker fint alt indholdet. De beskriver derefter meget fint relationen mellem artiklen og lærebogen. Og til sidst forklarer de fint hvordan artiklen påvirker deres eget projekt.
\section*{Abstract}
Gruppen beskriver fint og kort deres projekt og hvordan de vil arbejde med det.
\section*{Project purpose and restrictions}
I factor analysen forstår vi ikke helt hvorfor de har medtaget customer/admin i functionality da den bare skal beskrive hvilke der er, men ikke hvem den går til.\\
Derudover er den fint dækkende.\\
Designmæssigt må den gerne passe lidt bedre på siden så den går ned over sidetallet og bliver nemmere at overskue.
\section*{System Specifications}
\subsection*{Functional Requirements and Nonfuntional Requirements}
Her føler vi at begge skal være sat op punktvis da dette vil gøre det nemmere at få et overblik over disse krav. Hvilket føler vi føler man ikke får på den nuværende måde.
\subsection*{Use cases and Sequence diagrams}
I use cases skal :\\
Entry conditon skal rykkes op over flow of events i alle tilfælde.\\
De er ellers fint beskrivende som det skal være.\\ For øjets skyld så sørg for punkterene indexeres ens. \\ \\
I sekvens diagrammerne: \\
Vi forstår godt meningen, men de skal rettes til så diagrammet også viser at der bliver sendt noget tilbage til brugeren. F.eks. når kommer ind på product page skal der sendes noget tilbage til brugeren, som han så kan reagere på.\
\section*{System Design}
Her kunne de godt bare vise et billed af deres midlertidige side. Dette ville også gøre det nemmere at forholde sig til hvad det er de beskriver og vise deres flow på hjemmesiden.\\
Her mangler der også en video omkring produktet men det er forståeligt at den ikke er med grundet, hvor langt de er i projektet.\\
Ellers fint beskrevet.\\ 

\section*{Version Control}
Dækker fint deres ændringer, som man også kan se sammen med deres commit log.\\ Her kunne man måske beskrive kort, hvad det er for nogle bugs der er blevet fundet.
\section*{Project Cooperation}
Der bliver fint beskrevet hvordan arbejdet i gruppen fungerer. Hvad de har af planer med næste møde og hvad de snakkede om til mødet.\\  
Gruppen burde prøve at læse nogle af disse sætninger højt da et par af dem er relativt lange.\\
\end{document}