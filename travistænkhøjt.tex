\subsection*{Tænk højt øvelser}
Henning: Find det spot nærmest dig.\\
Umiddelbart kan jeg se at jeg er den store pil, og lige ved siden af den store pil er der en lille pil, så den vil jeg prøve at trykke på, hov der åbnede et vindue hvor der står fælledparken.\\\\
Bøge: Find Bmxbutikkens marker på kortet\\
Jeg prøver at trykke på en rød først og kan se at det er en skatepark, så jeg prøver den blå pil og der var bmxbutikken.\\\\
Lars: Ændrer indstillingerne.\\
Der er 2 tandhjul det må være indstillinger, så der trykker jeg og så kan jeg trække frem og tilbage på slideren og se flere eller færre spots.\\\\

\subsection*{Referat fra møde med kunde}
Møde med Travis den 8/6\\
Vi mødtes med travis i bmxbutikken eller som deres butik på nørrebrogade hedder "pedal & Co" Her fremviste vi vores program til ham og snakkede om forbedringer og evt ekstra funktionalitet, vi diskuterede også hvilken slags spots der skulle op og om man evt. skulle holde nogle spots hemmelige for brugerene for at ikke opfordre små børn til at hoppe over hegn f.eks.\\
Han var rigtig tilfreds med resultatet af programmet indtil videre, og glædede sig til at det blev færdig, så han kunne få det ud til den danske bmx scene. Han havde dog også nogle ting, som han synes vi skulle tænke ekstra over. Han synes bla. ikke at hjemmesiden var helt inuitiv nok han kom med et eksempel med den lille gule mand og google street view, hvor at en erfaren bruger af google maps ved at man skal trække den gule mand ned på kortet, vil en bruger der ikke er vand til google maps ikke kende den funktionalitet. Derfor ville han gerne have en knap inde i infowindows med streetview, så en bruger af appen også trykke på en knap indei  infowindowet og se googles billeder af spottet.\\
Derudover snakkede vi også om at få brugere til at bruge hjemmesiden, hvor vi blev enige om at det var vigtigt at starte med en database med en del spots, simpelthen for at få brugerne til at bruge hjemmesiden. Derudover hvis der først er brugere på siden vil de forhåbentlig også selv uploade deres egne lokale spots og på den måde kan spotguiden udvide sig til at være national, da der kommer massere af bmx'ere og skatere fra Jylland til hovedstaden for at køre\\
